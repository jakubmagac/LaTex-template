% !TEX root = ../thesis.tex

\chaptermark{Úvod}
\phantomsection
\addcontentsline{toc}{chapter}{Úvod}

\chapter*{Úvod}

Od prvej stránky, ktorá uzrelo svetlo sveta už ubehlo presne 30 rokov. Bola vytvorená CERNom a je na nej popísaný projekt World Wide Web a ako ho používať. Na počiatku webu bolo obyčajný HTML dokument. Malo ísť o jazyk, ktorým bude možné štruktúrovať text a zdieľať informácie pomocou internetu.

Za 30 rokov však vývoj webu pokročil a to čo boli kedysi jednoduché dokumenty s inofrmáciami, sú dnes pokročilé aplikácie, ktoré dokážu interagovať s používateľom a meniť svoj obsah na základe jeho správania sa na stránke.

Vytváranie takýchto aplikácie by bolo veľmi komplikované použitím čistého HTML, CSS a Javascriptu. Developeri sa snažili kedysi tento problém riešiť použitím knižníc ako napríklad jQuery ale ani to nestačilo pre veľké aplikácie typu Facebook alebo Google.

Dnešné aplikácie sú preto vyvýjané pomocou frameworkov ako napríklad Angular, React, Vue, a pod. 

Táto práca sa bude venovať práci s frameworkom Vue. 
% \footnote{\url{https://moodle.fei.tuke.sk/pluginfile.php/27971/mod_resource/content/16/Instructions_v15.pdf}}.


\section*{Formulácia úlohy}

Cieľom práce je zostaviť frontend aplikácie pre firmu US Steel. Aplikácia by mala slúžiť ako simulátor nastavení linky teplej valcovne. Aplikácia je vyvýjaná pod záštitou firmy Digital League, na jej backende pobeží machine learning model, ktorý na základe dát, ktoré zhromaždila US Steel vythodnotí najvhodnejšie nastavenie linky pre dané parametre a následne odsimuluje priebeh práce na danej linke. Príprava Machine Learning modelu však nie je súčasť tejto práce.

V tejto práci sa budeme venovať frontendu tejto aplikácie. Pozrieme sa na good practices pri vývoji Vue aplikácie. Porovnáme rôzne metódy písania komponentov, dizajnové patterny a pozrieme sa na aspekty, ktoré dokážu najviac zamávať s performence frameworku Vue. 

