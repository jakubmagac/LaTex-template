% !TEX root = ../thesis.tex

\chaptermark{Úvod}
\phantomsection
\addcontentsline{toc}{chapter}{Úvod}

\chapter*{Úvod}
Strojové učenie je oblasť umelej inteligencie a informatiky, ktorá sa sústredí na spracovanie dát a využitie učiacich sa algoritmov.

Strojové učenie má veľa využití v priemysle. Od samoriadiacích aut, cez rozpoznanie hlasu pre Google asistenta, po návrhy tovaru, ktorý by sa vám mohol páčiť na Amazone.

Pomocou Strojového učenia vieme odhadovať výstupy, na základe poskytnutých dát. 

Jednou z najjednoduchších metód použitia strojového učenia je linárna regresia. Pomocou lineárnej regresie vieme vypočítať linérne rastúce výstupy z vstupov. Príkladom môže byť cena auta, pizze alebo v našom prípade cena nehnuteľnosti, ktorej predikcii sa budeme venovať v tejto práci.


\section*{Formulácia úlohy}
Hlavnou náplňou tejto práce bude vytvoriť softvér pre predikciu cien nehnuteľností. Táto úloha sa bude skladať z viacero podúloh.

1. Vytvoriť nástroj na tvorbu datasetu na odhad trhovej ceny nehnutelnosti. Vytvoríme web scrapper, ktorý nám zozbiera dáta z weboovej stránky predávajúcej nehnuteľnosti a uloží ich do CSV formátu s ktorým bude pracovať náš model. Dáta následne prečistíme, vyselektovaním dát, ktoré ovplyvňujú celkovú cenu nehnuteľnosti, doplníme chýbajúce dáta a celkovo ich predspracujeme.

2. Analyzovať a znázorniť vlastnosti získaných dátových vzoriek. Pomocou knižnice pandas vytvoríme k získaným dátam grafy, pre lepšiu vizualizáciu súvislostí dát a výslednej ceny.

3. Implementovať softvérové riešeni využívajúce strojové učenie na odhad trhovej ceny nehnuteľnosti. Pôjde konkrétne riešenie modelu strojové učenia. Vyskúšame viacero algoritmov a parametrov a porovnaním nájdeme najvhodnejšie riešenie nášho problému.

4. Vhodnými metódami analyzujte presnosť vytvoreného riešenia. 

5. V rámci zadania vypracujte používateľskú a sýstémovú príručku. Používateľská príručka bude obsahovať pokyny a možnosti súšťania skriptu s modelom strojovho učnia. Systémová príručka bude určena programátorom a bude obsahovať zoznam použitých funkcií a algoritmov pre prípad, ak by sa nejaký programátor rozhodol pokračovať v ďalšom vývoji. Táto príručka bude slúžiť ako dokumentácia projektu. 