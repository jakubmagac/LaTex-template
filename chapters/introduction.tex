% !TEX root = ../thesis.tex

\chaptermark{Úvod}
\phantomsection
\addcontentsline{toc}{chapter}{Úvod}

\chapter*{Úvod}
Webstránky. Pár rokov dozadu jednoduché textové dokumenty slúžiace pre výmenu informácii prostredníctvom internetu, v dnešnej dobe komplexné aplikácie schopné konkurovať aj aplikáicam určených pre desktop.

Za 30 rokov vývoja sme prešli viacerými metódami pre vytváranie webový aplikácii. Od jedoduchých HTML a CSS súborov, jQuery, po dnešné veľké Javascriptové frameworky ako React, Angular alebo Vue.

Keďže dnešné aplikácie typu Facebook alebo Spotify môžu dosahovať obrovské rozmery, musíme pri nich preto aplikovať spôsoby aby sme ich spravili efektívnejšie pre používateľov a zaistili čo najrýchlejšiu odozvu. 

As page load time goes from one second to 10 seconds, the probability of a mobile site visitor bouncing increases 123%
(https://www.thinkwithgoogle.com/marketing-strategies/app-and-mobile/mobile-page-speed-new-industry-benchmarks-load-time-vs-bounce/)

Bounce rate označuje percento návštevníkov, ktorý navštívili stránku a hneď následne ju opustili. (Tzn. nepreklikli sa na žiadnu ďalšiu podstránku).

Takýto Bounce rate následne aj ovplyvňuje SEO stránky. Čím nižší rate ma stránka, tým nižšie ju vyhľadávač zaradzuje vo svojích výsledkoch.
\footnote{\url{https://www.websitebuilderexpert.com/building-websites/website-load-time-statistics/}}.

Táto práca má skúmať aspekty, ktoré dokážu ovplyvniť rýchlosť aplikácie. 

Porovnáme, ktoré su najpodstatnejšie, a ktoré budú mať najmenší vplyv.
% \footnote{\url{https://moodle.fei.tuke.sk/pluginfile.php/27971/mod_resource/content/16/Instructions_v15.pdf}}.


\section*{Formulácia úlohy}

Prvým cieľom teoretickej časti práce je popísať metódy, ktorými vieme najviac ovplyvniť rýchlosť aplikácie. Pozrieme sa na metódy ako Code-splitting, CDNs, Load balancing a podobne.

Druhým cieľom teoretickej časti je tieto metódy porovnať, a zistiť ktoré z nich budú mať najväčší vplyv pre rýchlosť aplikácie.

Praktická časť tejto práce je zostaviť frontend aplikácie pre firmu US Steel. Aplikácia by mala slúžiť ako simulátor nastavení linky teplej valcovne. Aplikácia je vyvýjaná pod záštitou firmy Digital League, na jej backende pobeží machine learning model, ktorý na základe dát, ktoré zhromaždila US Steel vythodnotí najvhodnejšie nastavenie linky pre dané parametre a následne odsimuluje priebeh práce na danej linke. Príprava Machine Learning modelu však nie je súčasť tejto práce. V tejto práci sa budeme venovať frontendu aplikácie. Aplikácia bude implementovan v jednom z najpoužívajnejších frontendových frameworkov - Vue. Konkrétne použijeme nadstavbu Vue s názvom Nuxt.
