% !TEX root = ../thesis.tex

\chapter{Implementácia modelu}


\section{Zber dát}
\subsection{Beautiful Soup}
Skrpity pre vytváranie datasetov sme riešili pomocou knižnice Beautiful Soup [6]. Knižnicu sme vybrali z dôvodu,aby sme nemuseli používať viacero programovacích jazykov. A keďže už riešime implementáciu modelov v Pythone, tak sme zvolili práve túto knižnicu. Knižnica slúži na vyťahovanie dát z HTML a XML súborov.

\subsection{Tvorba datasetov}
Pre tvorbu datasetov sme sa rozhodli použiť slovenský portál s realitami www.topreality.sk. Rozhodli sme sa vytvoriť viacero typov datasetov. Datasety sú delené na datasety pre domy a datasety pre byty. A následne sú ešte delené na predaj alebo prenájom nehnuteľností.


\subsection{Scripty}
Prvým scriptom pri tvorbe datasetov je script get-links. Tento script slúži na uloženie všetkých podstránok zo stránky topreality, potrebných pre datasety. Tento script rozdelí linky do štyroch súborov: predaj-byty.csv, prenajom-byty.csv, predaj-domy.csv a prenajom-domy.csv.

\section{Vizualizácia dát}
\subsection{Matplotlib}

\section{Implamentácia modelu}
\subsection{Porovnania a výber najvhodnejšieho modelu}

\section{Analýza presnosti riešenia}
\subsection{}

% \label{methodology}

% Syntetická časť opisuje metódy použité na syntézu riešenia a opisuje syntézu samotného riešenia (zvyčajne je to návrh/implementácia softvérového resp. hardvérového riešenia), pričom sa opiera o~závery analytickej časti práce. Začína od toho, ako sa bude riešenie používať: najdôležitejšie scenáre používania a používateľské rozhranie, ktoré bude tieto scenáre efektívne podporovať. Až potom je na rade vnútorná architektúra alebo použité technológie. Syntetická časť tvorí zvyčajne ½ jadra práce.

% Syntetickú časť práce vhodne rozdeľte do kapitol a pomenujte ich podľa toho, čomu sú venované.

% \begin{itemize}
%   \item v~knihe \cite{book} autor prezentuje naozaj odvážne myšlienky
%   \item nemenej zaujímavé výsledky publikuje ďalší autor v~článku \cite{article} 
%   \item v~konferenčnom príspevku \cite{conference} sú uvedené tiež zaujímavé veci
%   \item \LaTeX{}\footnote{\url{https://www.latex-project.org/}} je typografický jazyk
% \end{itemize}



% \begin{lstlisting}[language=C,caption={Program, ktorý pozdraví celý svet}]
%   #include <stdio.h>
%   int main() {
%       /* Print Hello, World! */
%       printf("Hello, World!\n");
%       return 0;
%   }
%   \end{lstlisting}


  
% \begin{table}[!ht]
% 	\caption{Country list}\label{t:1}
% 	\smallskip
% 	\centering

% 	\begin{tabular}{ |p{3cm}||p{3cm}|p{3cm}|p{3cm}|  }
% 		\hline
% 		\multicolumn{4}{|c|}{Country List} \\
% 		\hline
% 		Country Name or Area Name& ISO ALPHA 2 Code &ISO ALPHA 3 Code&ISO numeric Code\\
% 		\hline
% 		Afghanistan & AF & AFG & 004\\
% 		Aland Islands & AX & ALA & 248\\
% 		Albania & AL & ALB & 008\\
% 		Algeria & DZ & DZA & 012\\
% 		American Samoa & AS & ASM & 016\\
% 		Andorra & AD & AND & 020\\
% 		Angola & AO & AGO & 024\\
% 		\hline
% 	\end{tabular}
% \end{table}
