% !TEX root = ../thesis.tex

\chapter{Analytická časť}

\section{Vue}
Vue je progresívny frontendovy framework pre vytváranie použivateľských rozhraní (https://vuejs.org/v2/guide/). Pre účely tohto projektu sme zvolili Vue, keďže je najjednoduchší na naučenie (v porovnaní s Reactom a Angularom). Štruktúra Vue kódu je veľmi podobná s klasickou HTML, CSS a Javascript syntaxou. Doplnená o pár direktív Vue. Napríklad template tagy pre HTML časť. Vue taktiež na rozdiel od napríklad Reactu dovoľuje písať scooped CSS, nie je potrebná implementácia externých knižníc ako napríklad Styled Components, aby sme sa vyhli prepísavaniu CSS štýlov.


\subsection{Nuxt}
Nuxt je open source framework postavený na Vue. Nuxt sa nám snaží uľahčiť bežné problémy na ktoré narazíme pri vývoji appiek vo Vue. Jednou z nich je napríklad Automatický Router. Ak by sme chceli spravovať podstránky v klasickom Vue, museli by sme nastaviť súbor vue-router. Nuxt to rieši za nás a tento súbor generuje automaticky z jeho súborovej štruktúry. 

Základná inštalácia Nuxt projetku nám taktiež umožňuje pridať do projektu závislosti a vygeneruje nám súborovú štruktúru. Typesript pre statické typovanie, výber package manageru, UI knižnicu (Vuetify), Nuxt moduly (Axios, PWA, Content - git based headless CMS), Linting tool, Testing framework (Jest, Ava, WebdriverIO, Nightwatch), SSR/ SPA, Deployment Target (Server/ Static), Development tools,