% !TEX root = ../thesis.tex

\chapter{Analytická časť}

Analytická časť záverečnej práce analyzuje existujúce podobné prístupy k~riešeniu stanoveného problému. Autor práce musí uviesť v~tejto časti existujúce prístupy a riešenia, pričom musí zaujať stanovisko k~týmto prístupom a riešeniam a opísať ich výhody a nedostatky. Prevažne v~tejto časti autor používa odkazy na použité zdroje. Autor v~analýze nepreberá odseky z~cudzích prác ale uvádza prevažne vlastné postoje podložené odkazmi na literatúru. Analytická časť práce by teda nemala byť len povrchným prepisom základných informácií z~Wikipédie alebo zo stránok opisovaných nástrojov. Je potrebné aby bola analýza podporená aj experimentmi ak to umožňuje téma práce (napr. vyskúšam softvér). Vďaka popisu existujúcich riešení autor pochopí problematiku, viac sa nad riešeniami zamyslí, usporiada si ich, zistí ich kladné a záporné vlastnosti, z~čoho potom postupne vyplynie návrh vlastného riešenia v~syntetickej časti. Analytická časť tvorí zvyčajne ¼ jadra práce.

Analytickú časť je možné rozdeliť na niekoľko kapitol, ktoré budú venované rôznym analyzovaným témam. Názvy kapitol majú zodpovedať tomu, čo je v~kapitole opisované. Napríklad ak v~práci analyzujete súčasný stav v~oblasti medzigalaktických letov, namiesto všeobecného názvu "`Analýza súčasného stavu"' by mal byť použiťý názov analyzovanej témy --- "`Medzigalaktické lety"'.

% lorem ipsum
\section{Nuxt}


\section{Vuetify}
\begin{itemize}
    \item v~knihe \cite{book} autor prezentuje naozaj odvážne myšlienky
    \item nemenej zaujímavé výsledky publikuje ďalší autor v~článku \cite{article} 
    \item v~konferenčnom príspevku \cite{conference} sú uvedené tiež zaujímavé veci
    \item \LaTeX{}\footnote{\url{https://www.latex-project.org/}} je typografický jazyk
\end{itemize}

Given a set of numbers, there are elementary methods to compute its \acrlong{gcd}, which is abbreviated \acrshort{gcd}. This process is similar to that used for the \acrfull{lcm}.

\subsection{Donec vehicula consequat}
\blindtext

\begin{figure}[!ht]
    \centering
    \includegraphics[width=.9\textwidth]{figures/tugboat}
    \caption{\LaTeX{} Friendly Zone \label{o:latex_friendly_zone}}
\end{figure}

\subsection{Nullam in mauris consectetur}
\blindtext

\begin{lstlisting}[language=C,caption={Program, ktorý pozdraví celý svet}]
#include <stdio.h>
int main() {
    /* Print Hello, World! */
    printf("Hello, World!\n");
    return 0;
}
\end{lstlisting}


\subsection{Vestibulum tristique elementum varius}
\blindtext

\begin{table}[!ht]
	\caption{Country list}\label{t:1}
	\smallskip
	\centering

	\begin{tabular}{ |p{3cm}||p{3cm}|p{3cm}|p{3cm}|  }
		\hline
		\multicolumn{4}{|c|}{Country List} \\
		\hline
		Country Name or Area Name& ISO ALPHA 2 Code &ISO ALPHA 3 Code&ISO numeric Code\\
		\hline
		Afghanistan & AF & AFG & 004\\
		Aland Islands & AX & ALA & 248\\
		Albania & AL & ALB & 008\\
		Algeria & DZ & DZA & 012\\
		American Samoa & AS & ASM & 016\\
		Andorra & AD & AND & 020\\
		Angola & AO & AGO & 024\\
		\hline
	\end{tabular}
\end{table}


\section{Phasellus id pretium neque}
\blindtext

\blindtext
